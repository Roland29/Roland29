\documentclass[%singlesided,
    doublesided,
    paper=a4,
    fontsize=10pt
]{static/my-resume}



%%%%%%%%%%%%%%%%%%%%%%%%%%%%%%%%%%%%%%%%%%%%%%%%%%%%%%%%%%%%%%%%%%%%%%%%%%%%%%%%
% set geometry
%%%%%%%%%%%%%%%%%%%%%%%%%%%%%%%%%%%%%%%%%%%%%%%%%%%%%%%%%%%%%%%%%%%%%%%%%%%%%%%%

\setlength\highlightwidth{8cm}
\setlength\headerheight{4cm}            % note that margintop gets added to this value, i.e. the header bar is 5cm
\setlength\marginleft{1cm}
\setlength\marginright{\marginleft}      % needs to be 1.5 times to be actually equal. why?
\setlength\margintop{1cm}
\setlength\marginbottom{1cm}


%%%%%%%%%%%%%%%%%%%%%%%%%%%%%%%%%%%%%%%%%%%%%%%%%%%%%%%%%%%%%%%%%%%%%%%%%%%%%%%%
% FONTS
%%%%%%%%%%%%%%%%%%%%%%%%%%%%%%%%%%%%%%%%%%%%%%%%%%%%%%%%%%%%%%%%%%%%%%%%%%%%%%%%

\RequirePackage{fontspec}
\setmainfont{Carlito}


%%%%%%%%%%%%%%%%%%%%%%%%%%%%%%%%%%%%%%%%%%%%%%%%%%%%%%%%%%%%%%%%%%%%%%%%%%%%%%%%
% COLORS
%%%%%%%%%%%%%%%%%%%%%%%%%%%%%%%%%%%%%%%%%%%%%%%%%%%%%%%%%%%%%%%%%%%%%%%%%%%%%%%%

\colorlet{highlightbarcolor}{lightgray}
\colorlet{headerbarcolor}{darkgray}

\colorlet{headerfontcolor}{white}
\colorlet{accent}{awesome-red}
\colorlet{heading}{black}
\colorlet{emphasis}{black}
\colorlet{body}{black}


%%%%%%%%%%%%%%%%%%%%%%%%%%%%%%%%%%%%%%%%%%%%%%%%%%%%%%%%%%%%%%%%%%%%%%%%%%%%%%%%
% set document
%%%%%%%%%%%%%%%%%%%%%%%%%%%%%%%%%%%%%%%%%%%%%%%%%%%%%%%%%%%%%%%%%%%%%%%%%%%%%%%%


\begin{document}

    \name{Roland Paire}
    \tagline{Développeur senior avec une expertise en Node.js, React et AWS. \\+ 8 ans d'expériences, sur des projets variés (web app, mobile app, Saas, ...).\\ Passioné par les nouvelles technologies et les utiliser pour résoudre des problèmes du monde réel.}
    \photo[round]{static/profilePicture.png}{\dimexpr \headerheight-\marginbottom}   % make photo exactly match the header with margintop/marginright/marginbottom as margin

    \makeheader

    \highlightbar{

        \section{Contact}

        \email{roland.paire@gmx.fr}
        \phone{+33688032937}
        \vspace{0.5em}
        \github{@Roland}{https://github.com/Roland29}
        \linkedin{Roland Paire}{https://www.linkedin.com/in/roland-paire/}

        \section{Skills}

        \skillsection{Programming}
        \skill{Javascript}{5}
        \skill{NodeJs}{5}
        \skill{React}{3}
        \skill{NoSql}{4}
        \skill{AWS}{3}
        \skill{Typescript}{4}
        \skill{graphQl}{2}
        \skill{git}{4}

        \vspace{0.5em}
        \skillsection{Operating Systems}
        \skill{MacOS}{5}
        \skill{Linux}{4}
        \skill{Windows}{3}

        \vspace{0.5em}
        \skillsection{Software \& Tools}
        \skill{JIRA}{4}
        \skill{Docker}{3}
        \skill{Webstorm}{4}

        \vspace{0.5em}
        \skillsection{Languages}
        \skill{Français}{5}
        \skill{Englais}{4}
        \bigskip

    }
    \mainbar{
        \section{About this template}
        Section are set in bold face. An optional parameter of \texttt{\textbackslash section} takes a symbol to add in front of the text. This option is used in the jobs and education sections below.

        \section[\faGears]{Work history}
        \job{09/2021 - 01/2023}
        {Heyday by Hootsuite, Montréal Canada}
        {Senior Software Developer}
        {Développement d'un chatbot pour le e-commerce.
        Aide à la conception de l'architecture pour passer d'un monolithe vers des micro-services en serverless.
        Conception d'architecture micro-services basé sur le framework Nestjs. (nodejs, dynamodb, aws, typescript)
            Utilisation de AWS.
            Travaillé sur le développement du service de Single Sign On.Développement d'un chatbot pour le e-commerce. Aide à la conception de l'architecture pour passer d'un monolithe vers des micro-services en serverless. Conception d'architecture micro-services basé sur le framework Nestjs. (nodejs, dynamodb, aws, typescript) Utilisation de AWS. Travaillé sur le développement du service de Single Sign On.}
        \job{04/2020 - 11/2021}
        {Heyday,\\Montréal Canada}
        {Back-end Developer (freelance)}
        {Développement d'un chatbot, j'ai participé à l'intégration d'API tierce comme:
        - Prestashop
        - Magento
        - Salesforce

        J'ai travaillé sur l'intégration avec des canaux de communication:
        - WhatsApp
        - KakaoTalk (sud Coréen)
            - messenger
            - instagram
            - gbm
            - email}

        \section[\faMortarBoard]{Education}
        \job{01/2019 - 12/2019}
        {University, City}
        {Bachelor of Something}
        {}

        \section{Achievements, honours and awards}
        \achievement{My first achievement}
        \achievement{My second achievement}

        \section{General Skills}
        \smallskip % additional skip because tag outlines use up space
        \tag{Tag 1}
        \tag{Tag 2}
        \tag{and}
        \tag{another tag}
        \tag{some more tags}
        \tag{yet another one}
        \tag{tags flow over}
        \tag{to the next line}
        \tag{if necessary}

        \medskip
        Tags must be ordered by hand with newlines to get a nice layout, especially for long tags.

        \section{Wheel Chart}
        % This is taken from AltaCV
        % see https://github.com/liantze/AltaCV for details
        \wheelchart{1.5cm}{0.5cm}{% outer and inner diameter
            6/8em/accent!20/Sleep,          % comma-separated list of
            8/8em/accent!40/Daytime job,    % fraction of 24 / line length / color / label
            2/8em/accent!80/Training,          % here, the color is shades of the accent color
            3/8em/accent!60/Recovering from fighting criminals,
            5/8em/accent/Being Batman
        }
    }
    \makebody
    \clearpage


    \pagestyle{highlightmain}

% The highlightbar needs to be filled to display mainbar contents correctly in singlesised mode
% For an empty highlightbar, fill with empty space
    \highlightbar{\hfill}
    \mainbar{

        \section{Another section}

        This page uses the page style \texttt{highlightmain} which shows the highlight bar (gray) and the main part (white background) but omits the header.
        The default page style is \texttt{headerhighlightmain} with all three elements.
        If you don't want header, nor highlight bar, use page style \texttt{\textbackslash pagestyle\{empty\}}.
        \medskip
        Neither main, nor highlight bar must be filled to make this template work.
        It is possible to use a page style with the highlight bar but leave it empty by setting an empty highlightbar \texttt{\textbackslash highlightbar\{\}}.

        \vspace{0.5em}
        \subsection{Subsection 1}
        Demonstrate subsections.

        \subsection{Subsection 2}
        Subsection are also bold face but a smaller font then section. They also omit the rule.


    }
    \makebody


    \clearpage
    \pagestyle{empty}

    \section{Publications}
    \pubforcefullwidth

    Demonstrate what an \texttt{\textbackslash pagestyle\{empty\}} page looks like.
    Also show off the macros for publications that uses small icons for authors, date, journal and links.

    Achieving a good looking spacing can be tricky. For empty pagestyles where the full width is available use \texttt{\textbackslash pubforcefullwidth} to force the publoication list to take up all the available space.
    The (relative) lengths reserved for date, journal and links can be set with the parameters \texttt{\textbackslash pubdatelength}, \texttt{\textbackslash pubjournallength} and \texttt{\textbackslash publinklength} as in \texttt{\textbackslash setlength\{\textbackslash pubdatelength\}\{0.15 \textbackslash linewidth\}}.
    \bigskip

    \publication
    {The turbulent gas structure in the centers of NGC~253 and the Milky Way} % Title
    {\textbf{N. Krieger}, A. Bolatto, E. Koch, A. Leroy, E. Rosolowsky, F. Walter, A. Wei\ss, D. Eden, R. Levy, D. Meier, E. Mills, T. Moore, J. Ott, Y. Su, S. Veilleux} % Authors
    {2020} % Year
    {The Astrophysical Journal Vol. 899, Issue 2, id.158} % Journal
    {\ADS{https://ui.adsabs.harvard.edu/abs/2020ApJ...899..158K}, \arXiv{https://arxiv.org/abs/2008.02518}} % ADS & arxiv links

    \publication
    {The molecular ISM in the Super Star Clusters of the starburst NGC253} % Title
    {\textbf{N. Krieger}, A. Bolatto, A. Leroy, R. Levy, E. Mills, D. Meier, S. Veilleux, F. Walter, A. Wei\ss} % Authors
    {2020} % Year
    {The Astrophysical Journal Vol.897, Issue 2, id.176} % Journal
    {\ADS{https://ui.adsabs.harvard.edu/abs/2020ApJ...897..176K}, \arXiv{https://arxiv.org/abs/2006.08262}} % ADS & arxiv links

    \publication
    {The Molecular Outflow in NGC\,253 at a Resolution of Two Parsecs} % Title
    {\textbf{N. Krieger}, A. Bolatto, F. Walter, A. Leroy, L. Zschaechner, D. Meier, J. Ott, A. Wei\ss, E. Mills, S. Veilleux, M. Gorski} % Authors
    {2019} % Year
    {The Astrophysical Journal Vol.881, Issue 1, article id. 43, 20 pp} % Journal
    {\ADS{https://ui.adsabs.harvard.edu/abs/2019ApJ...881...43K}, \arXiv{https://arxiv.org/abs/1907.00731}} % ADS & arxiv links

\end{document}